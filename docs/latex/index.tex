\chapter{Trabalho prático }
\hypertarget{index}{}\label{index}\index{Trabalho prático@{Trabalho prático}}
O objetivo deste trabalho é implementar um programa que peça ao utilizador 18 números inteiros e os guarde num vetor, para posteriormente providenciar forma de calcular algumas estatísticas ou fazer operações sobre esses valores. Os valores pedidos devem estar compreendidos entre 1 e 26. Deve ser feita a VALIDAÇÃO DE ENTRADA!

Após terem sido pedidos os valores, deve ser mostrado um menu ao utilizador que lhe permita calcular cada uma das estatísticas referidas em baixo, exatamente pela ordem colocadas neste enunciado. Depois de se escolher uma opção, o resultado deve ser mostrado no ecrã, e o menu deve voltar a ser exibido. As funcionalidades mínimas a disponibilizar são as seguintes\+: 1 -\/ Identificação do máximo de todos os elementos do vetor; 2 -\/ Construção de uma matriz 4 por 18, em que cada linha é composta pelo vetor lido; 3 -\/ Cálculo da tangente (tan) da primeira metade dos elementos no vetor; 4 -\/ Devolução da soma dos valores do vetor que são divisíveis por três; 5 -\/ Cálculo da subtração da primeira metade dos elementos no vetor com os da segunda metade (dá um vetor com metado do tamanho); 6 -\/ Devolução do vetor ordenado por ordem decrescente. Uma versão mais elaborada do projeto deve exibir adicionalmente as seguintes características e funcionalidades\+: 1 -\/ Leitura de um novo vetor, cálculo e devolução do produto interno; 2 -\/ Apresentação da decomposição em números primos dos números impares no vetor inicial; 3 -\/ Leitura de um novo vetor 1x18, cálculo e devolução da matriz 18x18 resultante do produto do vetor inicial com o novo vetor gerado; 4 -\/ Cálculo do determinante da matriz referida no ponto anterior; 5 -\/ O programa apresenta adicionalmente uma página de ajuda, acessível como sendo a entrada 7 no menu. 6 -\/ O programa mostra alguma ajuda quando é executado a partir da linha de comandos com a flag --help. 